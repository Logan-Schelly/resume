\documentclass{article}
\usepackage[utf8]{inputenc}
\usepackage[letterpaper, margin=.5in]{geometry}
\usepackage[parfill]{parskip} % Remove paragraph indentation
\usepackage{enumitem} % Modify list layouts
\usepackage{tabularx} % Tables the same width as the page.
\usepackage{scrextend} % Add margin environment
\usepackage{amsfonts} % Math symbols
\usepackage{calc} % widthof and length operations (like subtraction)
\usepackage[colorlinks, urlcolor=blue]{hyperref}
\usepackage{multicol}
\usepackage{comment}
\pagestyle{empty} % Suppress page numbers
\setlength{\multicolsep}{0pt} %Prevent multicol from hogging vertical whitespace


%%%%%%%%%%%%%%%%%%%%%%%%%%%%%%%%%%%%%5555
%My Macros: TODO: move these into a .sty sheet.
%%%%%%%%%%%%%%%%%%%%%%%%%%%%%%%%%%%%%%%%%%%
%Compact lists TODO: just set these to default.
\newenvironment{compactItemize}{
  \begin{itemize}[itemsep=0ex, parsep=0ex, partopsep=0ex, topsep= -7pt]
}{
  \end{itemize}
}


%Fill with line at text level.
\newcommand{\lineFill}{\leaders\hbox{\rule[.5ex]{1pt}{.1pt}}\hfill}

%Technical Skill Listings
\newcommand{\technicalSkill}[3]{
  \makebox[15ex][l]{#1 \hfill} 
  \makebox[\widthof{Proficient, but rusty}+2ex][l]{#2}
  #3
}
\newenvironment{technicalSkillWithSubskills}[3]{
  \newcommand{\subskill}[1]{} %Don't put subskills -- this is the simple version
  \technicalSkill{#1}{#2}{#3}
  %\smallskip
  %\begin{multicols}{4}
  %\begin{itemize}[parsep=0ex, partopsep=0ex, topsep=-.7ex, itemsep=-.5ex]
  }{
  %\end{itemize}
  %\end{multicols}
  %\medskip
}

%Job info Listings
\newcommand{\jobInfo}[4]{
  \begingroup
  \setlength{\tabcolsep}{0ex}
  \begin{tabularx}{\linewidth}{X r}
    #1 & % position 
    #2\\ % time range
    #3 & % company
    #4   % location
  \end{tabularx}%
  \endgroup%
}
\newcommand{\degree}[4]{
  % Use same layout as jobInfo
  % Order is degree, grad date, institution, GPA)
  \jobInfo{\textbf{#1}}{#2}{#3}{#4}
}
\newcommand{\project}[2]{
  \begingroup
  \setlength{\tabcolsep}{0ex}
  \begin{tabularx}{\linewidth}{X r}
    #1 & % project name
    #2\\ % time range
  \end{tabularx}%
  \endgroup%
}

\begin{document}

\begin{center}
{\Large Logan Schelly} %Make a \fullName command.

(951)\phantom{-}692-8802 %Make a \phoneNumber{}{}{} command \
$\bullet$ %make a separator commmand.                        |-> Make an essential contact info command.
idyllogan@verizon.net %Make an \email command.              /
\end{center}
\begin{comment}
\textbf{Objective}
\smallskip
\hrule
I am a recent graduate.
My ultimate career goal would be to work on a project like Intel's Math Kernel Library.
For the time being I am looking for an entry-level job in software, data, or (if possible) high performance computing.
\end{comment}
\textbf{Education}
\smallskip
\hrule

\degree{Bachelors of Science in Mathematics}{April 2020}{Brigham Young University}{}%GPA: 3.17 out of 4.0}
%
\begin{itemize}[labelindent=2ex, itemsep=0ex, parsep=0ex, partopsep=0ex, topsep= -.7ex]%Need to undo the parskip from usual start
  \item Applied and Computational Mathematics Emphasis (ACME)
  \item Computer Science Minor
  \item Recognized for outstanding performance in Mathematics in 2020 and 2018
  \item Routinely identified and reported the most typos in developing ACME textbooks
\end{itemize}

\begin{comment}{2ex}

\textbf{Course Work and Topics}
  \begin{itemize}[labelindent=2ex, parsep=0ex, partopsep=0ex, topsep=-.7ex]
    \item Fundamentals of Mathematics
      \begin{multicols}{4}
        \begin{itemize}[labelindent=2ex, parsep=0ex, partopsep=0ex, topsep=-.7ex]
          \item Set Theory
          \item Logic
          \item Proof Techniques
          \item Relations
          \item Functions
          \item Induction
          \item Cardinality
          \item Number Theory
        \end{itemize}
      \end{multicols}
    \item Linear Algebra
      \begin{multicols}{3}
        \begin{itemize}
          \item Solving Linear Systems
          \item Matrix Algebra
          \item Determinants
          \item Vector Spaces
          \item Eigenvectors
          \item Inner Product Spaces
          \item Quadratic Forms
          \item Singular Value Decomposition
        \end{itemize}
      \end{multicols}
    \item Calculus of Several Variables
      \begin{multicols}{3}
        \begin{itemize}
          \item Quadric Surfaces
          \item Vector Functions
          \item Partial Derivatives
          \item Multiple Integrals
          \item Vector Calculus
        \end{itemize}
      \end{multicols}
    \item Differential Equations
      \begin{multicols}{2}
        \begin{itemize}
          \item First Order Differential Equations
          \item Second Order Linear Differential Equations
          \item Series Solutions of Second Order Equations
          \item Laplace Transformation
          \item Systems of First Order Linear Equations
          \item Numerical Methods
        \end{itemize}
      \end{multicols}
    \item Theory of Analysis
      \begin{multicols}{2}
        \begin{itemize}
          \item Properites of the Real Numbers
          \item Sequences and Series
          \item Topology of $\mathbb{R}$
          \item Limits and Continuity of Functions
          \item Derivatives
          \item Sequences and Series of Functions
          \item Riemann Integration
        \end{itemize}
      \end{multicols}
    \item Mathematical Analysis
      \begin{multicols}{2}
        \begin{itemize}
          \item Abstract Vector Spaces
          \item Linear Transformations and Matrices
          \item Inner Product Spaces
          \item Spectral Theory
          \item Metric Space Topology
          \item Fr\'echet Differentiation
          \item Contraction Mappings
          \item Daniell-Lebesgue Integration
          \item Calculus on Manifolds
          \item Complex Analysis
          \item Spectral Calculus
          \item Iterative Methods for Linear Systems
       \end{itemize}
      \end{multicols}
     \item Algorithm Design and Optimization
      \begin{multicols}{2}
        \begin{itemize}
          \item Measuring Algorithm Complexity
          \item Data Structures
          \item Combinatorial Optimization
          \item Probability
          \item Probabilistic Sampling and Estimation
          \item Random Algorithms
          \item Harmonic Analysis
          \item Polynomial Approximation and Interpoloation
          \item Unconstrained Optimization
          \item Linear Optimization
          \item Nonlinear Constrained Optimization
          \item Convex Analysis and Optimization
          \item Dynamic Optimization
          \item Stochastic Dynamic Optimization
       \end{itemize}
      \end{multicols}
    \item Modeling with Uncertainty and Data
      \begin{multicols}{2}
        \begin{itemize}
          \item Markov Chains
          \item Classical Inference
          \item Hypothesis Testing
          \item Regression and Classification
          \item Bayesian Analysis
          \item Estimation in State Space Models
          \item Machine Learning
          \item Unsupervised Methods
          \item Graphical and Latent Variable Models
          \item Kernel Methods
          \item Tree-Based Methods
        \end{itemize}
      \end{multicols}
    \item Modeling with Dynamics and Control
      \begin{multicols}{2}
        \begin{itemize}
          \item Existence and Uniqueness Theorem
          \item Stability Theory
          \item Bifurcation Theory
          \item Partial Differential Equations
          \item Calculus of Variations
          \item Euler's Equation
          \item Hamilton's Principle
          \item Noether's Theorem
          \item Optimal Control
          \item Pontryagin's Maximum Principle
          \item Linear Quadratic Regulators
        \end{itemize}
      \end{multicols}
  \end{itemize}
\end{comment}


\textbf{Skills}
\smallskip
\hrule
\textbf{Programming Languages}
\begin{itemize}[parsep=1ex, partopsep=0ex, topsep=-.7ex, itemsep=0ex]
  \item \technicalSkill{Python}{Very Comfortable}{used in 8 lab classes and 4 lecture classes}
    \begin{itemize}
      \item Amoung other things, I've used both the Selenium and pytest libraries in class projects.
    \end{itemize}
  \item \technicalSkill{JavaScipt}{Beginner}{self taught at \href{https://javascript.info}{javascript.info}, and listed as a \href{https://javascript.info/about}{contributor}}
    \begin{itemize}
      \item My contributions are mostly testcases using Mocha/Chai so people can check their answers to exercises.
    \end{itemize}
  \item \technicalSkill{Java}{Proficient, but rusty}{used extensively in 1 lecture class}
  \item \technicalSkill{C}{Proficient}{used in 2 lecture classes}
  \item \technicalSkill{C++}{Proficient, but rusty}{used in 3 lecture classes}
\end{itemize}

\textbf{Other Tools}
\begin{multicols}{2}
  \begin{compactItemize}
    \item \LaTeX -- Proficient
    \item Git -- Intermediate
    \item HTML -- Beginner
  \end{compactItemize}
\end{multicols}

\textbf{Soft Skills}
  \begin{center}
    Attention to Detail $\bullet$ Troubleshooting $\bullet$ Project Coordination
  \end{center}

\textbf{Work Experience}
\smallskip
\hrule

\jobInfo{Head Upper Division Tutor}{Provo, UT}{BYU Math Lab}{Sep 2014 -- April 2020}
\begin{compactItemize}
  \item Helped students on a first-come first-served basis.
  \item Tutored Linear Algebra, Multivariable Calculus, Differential Equations, and Mathematical Proof classes.
  \item Over my tenure, helped thousands of students find mistakes in their work, or identify points of misunderstanding.
  \item Conducted weekly meetings to help our team of 10-20 upper division tutors prepare for that week's concepts.
  \item Coordinated exam reviews, and often taught them.  10 to 200 students attended, depending on enrollment and subject.
  \item Overhauled the tutor handbook.
  \item Expanded the tutor application test to include a Mathematical Proof section.
\end{compactItemize}
\medskip
\textbf{Relevant School Projects %Of Interest
}
\smallskip
\hrule

\project{Math Lab Student Sign Up Analysis}{Fall 2019 -- Winter 2020}
\begin{compactItemize}
  \item Consolidated sign-up data spread across 60+ Excel files.
  \item Used Pandas to analyze the almost 900,000+ instances of students signing up for tutor help.
  \item Wrote a Python program that used Selenium to scrape the historical enrollment data at \href{https://classschedule.byu.edu}{classschedule.byu.edu}.
  \item Identified busiest times of the week, and the topics students most often came in for help with.
  \item Advised scheduling more tutors in the mornings based on my findings.
\end{compactItemize}
\medskip
\begin{comment}
\project{HTTP Proxy}{Winter 2020}
\begin{compactItemize}
  \item C program that relayed user requests to end server, and relayed server responses to user.
  \item Used \texttt{regex.h} to verify that user requests met HTTP formatting requirements.
  \item Handled concurrent requests with a threadpool using \texttt{pthread.h} and \texttt{semaphore.h}.
\end{compactItemize}
\medskip

\project{DNS Stub Resolver}{Winter 2020}
\begin{compactItemize}
  \item Program interfaced with DNS servers to look up IP addresses associated with a web domain name.
        For example, it would figure out that the domain name www.example.com is associated with IP address 93.184.216.34.
  \item Formatted queries to DNS standards, sent the queries with UDP, and then decoded responses.
  \item Written C with \texttt{unistd.h}, \texttt{sys/socket.h}, \texttt{arpa/inet.h}, and \texttt{netinet/in.h}.
\end{compactItemize}
\medskip

\project{OpenMP with Mandelbrot Set}{Winter 2020}
\begin{compactItemize}
  \item Parallelized the \href{https://gist.github.com/andrejbauer/7919569}{Mandelbrot visualization code} posted on github by Andrej Bauer.
\end{compactItemize}
\medskip

\project{Tiny Shell}{Winter 2020}
\begin{compactItemize}
  \item Wrote a simple shell that could handle process creation, I/O redirection and pipelines, and process control.
  \item Used C with \texttt{unistd.h} and \texttt{signal.h}.
\end{compactItemize}
\medskip
\end{comment}
\project{Inverted Pendulum Control}{Winter 2019}
\begin{compactItemize}
  \item Modified the Python code from the CartPole-v1 environment of OpenAI's gym library.
  \item Updated from Euler's method to Runge-Kutta.
  \item Applied an LQR control scheme to keep the pendulum upright.
\end{compactItemize}
\medskip

\project{Android App -- Family History Map}{Summer 2018}
\begin{compactItemize}
  \item Wrote both the client and server in Java.
  \item Displayed family history data with a Google MapFragment.
  \item Implemented activities for log-in, map interaction, life event details, and app settings.
  \item Wrote the SQL commands that the server would use to store and retrieve user data.
\end{compactItemize}

\begin{comment}
  TODO:
  -Macro command for courses list.
  -Push macros into other file (lsResume.sty)
  -Projects:
    >

      \subsection*{Graph-Based Image Segmentation}
  \subsection*{SVD Image Compression}
  \subsection*{Facial Recognition with Eigenfaces}
  \subsection*{Applications of PageRank Algorithm}
  \subsection*{Drazin Inverse for Social Network Link Prediction}
  \subsection*{Jacobi, Gauss-Seidel, and Successive Over-Relaxation for Solving Linear Systems}
  \subsection*{Linked Lists}
  \subsection*{Binary Search Trees}
  \subsection*{K-D Trees for K-Nearest Neighbors}
  \subsection*{Breadth First Search and Six Degrees of Kevin Bacon}
  \subsection*{Random Speech Generation with Markov Chains}
  \subsection*{Simplex Method}
\end{comment}

\end{document}
