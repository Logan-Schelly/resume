\documentclass{article}
\usepackage[utf8]{inputenc}
\usepackage[colorlinks, urlcolor=blue]{hyperref}
\usepackage[letterpaper, margin=.5in]{geometry}

\pagestyle{empty}

% Block paragraphs, no indentation.
\setlength{\parindent}{0em}
\setlength{\parskip}{1.5em}

\begin{document}

\begin{center}
{\Large Logan Schelly}

(951)\phantom{-}692-8802
$\bullet$
idyllogan@verizon.net
\end{center}

\hrule
\bigskip
\noindent16 July 2020
\bigskip

To whom it may concern,
\bigskip

Over the course of my college education, I've come to enjoy both Math and Computer Science.
It's especially satisfying to see how careful algorithm design and modern hardware can cut through tedious computations like a hot knife through butter.
My ultimate career goal would be to work on a mathematics software library like Intel's MKL or NVIDIA's CUDA.
However, amoung other things, I need more exprerience with parallel computing and low-level programming.
That's why I'm looking for opportunities in High Performace Computing, and why I'm applying to work 
at KLA as a HPC Engineer.
While this position might be a bit of a stretch for me, it stretches me in exactly the kind of ways I want to grow.

That's not to say that I'm completely unqualified.
I believe that my Bachelor's education in Mathematics,
my experience with BYU's Applied and Computational Mathematics Emphasis (ACME),
and my minor in Computer Science have prepared me for this position.

For example, I'm accustomed to working in a Linux environment.
The ACME lab machines ran Redhat Linux and the Computer Science machines used KDE Plasma Linux.
When I finally bought a laptop, I installed Ubuntu Linux and I've been using it ever since.
I used an IDE for only two Computer Science classes.
For the rest of them I used a text editor and a Linux terminal.
Furthermore, my systems programming class required us to use C libraries to interact with the Linux kernel.
While I don't know everything about Linux, it's my default working environment;
I'm actually using the terminal, vim, and pdflatex to write this cover letter.

Additionally, I've become familiar with computer architecture concepts.
In my Computer Systems class, my professor liked to say we learned about computers ``from dirt":
we started learning about MOSFET transistors, and worked our way up through logic circuits,
sequential execution of machine code, and compilers.
Along the way we did accompanying labs where we programmed TI-MSP430 microcontrollers in both assembly and C.
I learned more about the Linux kernel and the logistics of virtual memory in my Systems Programming class.

I've also been indirectly introduced to shell scripting.
After I got tired of retyping compilation, execution, and output comparison commands in class labs,
I started writing my own makefiles that included an automatic output comparison.
Additionally, my Systems Programming professor often distributed the shell scripts the TAs would use for grading labs.
Every time I was working through a bug, I read through the script to see what \emph{exactly} his tests were doing to my program.
Perhaps that was his clever way of introducing me to shell scripting.

Additionally, I've been introduced to some of the tools of distributed and parallel computing.
My ACME classes had labs where we used IPython parallel and MPI for Python.
My computer science classes had projects where we used POSIX threads.

Overall, this position as a HPC Engineer seems like an excellent opportunity to exercise my 
problem-solving and analysis skills I've picked up through my Mathematics education,
and develop new skills in computing.
Thank you for your time and consideration.
I look forward to hearing back from you soon.

\bigskip
\noindent Logan Schelly
  
\end{document}
