\documentclass{article}
\usepackage[utf8]{inputenc}
\usepackage[colorlinks, urlcolor=blue]{hyperref}
\usepackage[letterpaper, margin=.5in]{geometry}

\pagestyle{empty}

\begin{document}

\begin{center}
{\Large Logan Schelly}

(951)\phantom{-}692-8802
$\bullet$
idyllogan@verizon.net
\end{center}

\hrule
\bigskip
\noindent25 June 2020
\bigskip

To whom it may concern,
\bigskip

Hello!  My name is Logan Schelly.
I'm a recent math graduate from BYU.
I'm looking for a job where I can contribute as an effective employee and team member
and where I can develop as a computer scientist and mathematician.
I think Qualtric's listed position as a Software Test Engineer might be that kind of job.

How would I be able to contribute as an employee and team member?
First, I already have some experience with software testing tools.
My computer science minor introduced me to tools like Java's JUnit
when I took a class where we built an Android app.
My applied math courses introduced me to tools like pytest and Selenium.
I ended up using both of those tools for my personal class project analyzing
when students come to BYU's math lab.
I used Selenium to scrape enrollment data and class syllabi, and I
used pytest to evaluate how my combination of 
Beautiful Soup and regex expressions did at parsing assigned problems and due dates
from html-based syllabi.
I've also gained experience with Mocha and Chai outside of school.
Since graduation, I've started learning JavaScript from the open-source guide at
\href{https://javascript.info}{javascript.info}.
Chapters have exercises that sometimes come with their own Mocha/Chai tests.
After the chapter on automated testing with Mocha, 
I started contributing my own tests for exercises that didn't have any.

Additionally, I think my background as a math major and tutor would help me contribute to Qualtrics.
I spent enormous amounts of time as a tutor ``debugging" student's work to find their mistakes.
This wasn't a skill I used only while tutoring.
In an effort to learn more effectively, I would often try to poke holes in the proofs of my math textbooks.
When I entered BYU's applied math program, this turned out to be useful for both me and my professors.
Since they were actively developing our textbooks, they asked us to report mistakes and typos to them.
They kept an online errata, and cited who had found each mistake, so I can tell you
that I was consistently the student who reported the most mistakes each semester.
If you have more questions about my contributions to the applied math textbooks, I'm sure 
Professor Jarvis (\href{mailto:jarvis@math.byu.edu}{jarvis@math.byu.edu}), 
Professor Evans (\href{mailto:ejevans@math.byu.edu}{ejevans@math.byu.edu}), or 
Professor Humpherys (\href{mailto:jhumpherys@gmail.com}{jhumpherys@gmail.com}) would be happy to explain how helpful I was.

Qualtrics seems like an excellent place to start my career and grow as a software and data professional.
I hope to meaningfully contribute as a Software Test Engineer, and I look forward to hearing back from you soon.

\bigskip
\noindent Thank you,\\
\noindent Logan Schelly
  
\end{document}
