\documentclass{article}
\usepackage[utf8]{inputenc}
\usepackage[colorlinks, urlcolor=blue]{hyperref}
\usepackage[letterpaper, margin=.5in]{geometry}

\pagestyle{empty}

% Block paragraphs, no indentation.
\setlength{\parindent}{0em}
\setlength{\parskip}{1em}

\begin{document}
\begin{center}
{\Large Logan Schelly}

(951)\phantom{-}692-8802
$\bullet$
idyllogan@verizon.net
\end{center}

\hrule
\bigskip
30 June 2020
\bigskip\\
Christine Lien and Andrew Baker\\
Derivita\\
9690 South 300 West\\
Suite 308\\
Sandy, UT 84070
\bigskip\\
Ms. Lien and Mr. Baker,
\medskip\\
Yesterday I learned about the openings at Derivita when Chris Coca, a former classmate and LinkedIn connection of mine,
shared a post by Ms. Lien.
I've worked as a math tutor for over five fulfilling years.
I've enjoyed a cycle of studying (and restudying) math concepts,
gaining new insights,
using these insights to present math in a more understandable and engaging way,
listening to student feedback and questions,
and then using this feedback and questions to identify specific topics to cycle back to.
It's a very satisfying process.
Lately, I've also been looking for opportunities to practice and develop my coding skills.
After reading Ms. Lien's post, I checked the company website and was interested to see a posting as an Implementer (also listed as a Question Coder).

This position excites me because I think it would be an excellent opportunity to not only grow as a programmer,
but also use the skills I developed and experience I've gained over the last several years.
This experience isn't just restricted to my time as a tutor.
Like Chris, I didn't just do the vanilla math program at BYU.
We both did the Applied and Computational Mathematics Emphasis (ACME).
The ACME curriculum was under active development my entire time at BYU,
so we were often studying a draft version of the textbook.
Our professors encouraged us to examine the text very critically,
and report any typos we found.
I'm sure both Chris and my professors will tell you I took this advice to heart;
I had a knack for finding and reporting errors.
Critically examining instructional mathematics material isn't the only experience
I can leverage to help you.
I've also completed a Computer Science minor.
Additionally, I've been picking up JavaScript this summer by working through the open-source tutorials at
\href{https://javascript.info}{javascript.info} 
(if you check out the GitHub project, you'll find I've even made some \href{https://github.com/javascript-tutorial/en.javascript.info/graphs/contributors?from=2014-10-26&to=2020-08-01&type=a}{modest contributions}).

Derivita's position as a Question Coder sounds like it aligns well with both my experience and my interests.
I'd love to talk about how I can use my time and abilities to help your team, your clients, your teachers, and your students.
\bigskip\\
Thank you for your time,\\
Logan Schelly


\end{document}
