\documentclass{article}
\usepackage[utf8]{inputenc}
\usepackage[colorlinks, urlcolor=blue]{hyperref}
\usepackage[letterpaper, margin=.5in]{geometry}

\pagestyle{empty}

% Block paragraphs, no indentation.
\setlength{\parindent}{0em}
\setlength{\parskip}{1.5em}

\begin{document}

\begin{center}
{\Large Logan Schelly}

(951)\phantom{-}692-8802
$\bullet$
idyllogan@verizon.net
\end{center}

\hrule
\bigskip
\noindent30 June 2020
\bigskip

To whom it may concern,
\bigskip

My first year back at BYU was rough.
I lost my scholarship, nearly went broke, and couldn't sleep well at night.
But, the one thing that kept me going was my job supporting and advising students at the BYU Math Lab.
Each time that I was able to help a student understand the topics from their course,
or I was able to help them figure out how to solve a problem,
my day got a little better.

Since I graduated in April, I've been looking for the next step in my career.
While I enjoy helping people with technical problems, I'd like to spend time outside academia.
Every math student wonders ``How can I use this?"
I want to work at a job where each day answers that question.
That's why I'm applying as a Technical Support Advisor at Domo.

I believe that this role is one where I can contribute to the company
while also growing as a mathematician, a computer scientist, and an advisor.
But, how would I contribute?

First, I think my experience as a math tutor will transfer well into a technical support role.
For more than 5 years, I have been helping students learn how to use math concepts to accomplish their academic goals.
Now I would be helping professionals learn how to use Domo's features to accomplish their business goals.
All the listening, communication, and troubleshooting skills I exercised as a tutor seem like a great fit for this position.

Second, I think my education will help me master the Domo platform and surrounding technologies.
My computer science minor obviously gave me solid experience with programming languages,
and my mathematics major was surprisingly helpful.
Since I did the Appplied and Computational Mathematics Emphasis (ACME),
for each lecture class I had an accompanying lab class
where I would use python to both code up the algorithms and solve the problems I had learned about in lecture.

I learned when I came back to BYU that a job that fits my skill set and allows me to help others is an opportunity I shouldn't let pass by.
This job seems like an opportunity I shouldn't let pass by.
I hope to hear from you in the coming weeks.

\bigskip
\noindent Thank you,\\
\noindent Logan Schelly
  
\end{document}
